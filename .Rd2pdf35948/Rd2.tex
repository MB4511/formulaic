\documentclass[a4paper]{book}
\usepackage[times,inconsolata,hyper]{Rd}
\usepackage{makeidx}
\usepackage[utf8]{inputenc} % @SET ENCODING@
% \usepackage{graphicx} % @USE GRAPHICX@
\makeindex{}
\begin{document}
\chapter*{}
\begin{center}
{\textbf{\huge Package `formulaic'}}
\par\bigskip{\large \today}
\end{center}
\begin{description}
\raggedright{}
\inputencoding{utf8}
\item[Title]\AsIs{Dynamic Generation and Quality Checks of Formula Objects}
\item[Version]\AsIs{0.0.7}
\item[Description]\AsIs{Many statistical models and analyses in R are implemented through formula objects. The formulaic package creates a unified approach for programmatically and dynamically generating formula objects. Users may specify the outcome and inputs of a model directly, search for variables to include based upon naming patterns, incorporate interactions, and identify variables to exclude. A wide range of quality checks are implemented to identify issues such as misspecified variables, duplication, a lack of contrast in the inputs, and a large number of levels in categorical data.  Variables that do not meet these quality checks can be automatically excluded from the model.  These issues are documented and reported in a manner that provides greater accountability and useful information to guide an investigation of the data.}
\item[Depends]\AsIs{R (>= 3.2.0)}
\item[URL]\AsIs{}\url{https://dachosen1.github.io/formulaic/index.html}\AsIs{}
\item[BugReports]\AsIs{}\url{https://github.com/dachosen1/formulaic/issues}\AsIs{}
\item[License]\AsIs{GPL-3}
\item[Encoding]\AsIs{UTF-8}
\item[LazyData]\AsIs{TRUE}
\item[RoxygenNote]\AsIs{7.1.1}
\item[Imports]\AsIs{data.table}
\item[Suggests]\AsIs{knitr,
rmarkdown,
testthat (>= 2.1.0),
covr}
\item[VignetteBuilder]\AsIs{knitr}
\end{description}
\Rdcontents{\R{} topics documented:}
\inputencoding{utf8}
\HeaderA{add.backtick}{Add backtick}{add.backtick}
%
\begin{Description}\relax
Function that add backticks to the input variables.
\end{Description}
%
\begin{Usage}
\begin{verbatim}
add.backtick(x, include.backtick = "as.needed")
\end{verbatim}
\end{Usage}
%
\begin{Arguments}
\begin{ldescription}
\item[\code{x}] Character value specifying the name of input parameters.

\item[\code{include.backtick}] specifies whether a backtick should be added.
Parameter values should be either 'all' or 'as.needed'
\end{ldescription}
\end{Arguments}
\inputencoding{utf8}
\HeaderA{create.formula}{Create Formula}{create.formula}
%
\begin{Description}\relax
Create formula is a tool to automatically create a formula object from a
provided variable and output names. Reduces the time required to manually
input variables for modeling. Output can be used in linear regression,
random forest, neural network etc. Create formula becomes useful when
modeling data with multiple features. Reduces the time required for modeling
and implementation :
\end{Description}
%
\begin{Usage}
\begin{verbatim}
create.formula(
  outcome.name,
  input.names = NULL,
  input.patterns = NULL,
  dat = NULL,
  interactions = NULL,
  force.main.effects = TRUE,
  reduce = FALSE,
  max.input.categories = 20,
  max.outcome.categories.to.search = 4,
  order.as = "as.specified",
  include.backtick = "as.needed",
  format.as = "formula",
  variables.to.exclude = NULL,
  include.intercept = TRUE
)
\end{verbatim}
\end{Usage}
%
\begin{Arguments}
\begin{ldescription}
\item[\code{outcome.name}] A character value specifying the name of the formula's
outcome variable. In this version, only a single outcome may be ed. The
first entry of outcome.name will be used to build the formula.

\item[\code{input.names}] The names of the variables with the full names delineated.
User can specify '.' or 'all' to e all the column variables.

\item[\code{input.patterns}] es additional input variables. The user may enter
patterns -- e.g. to e every variable with a name that es the pattern.
Multiple patterns may be ed as a character vector. However, each pattern
may not contain spaces and is otherwise subject to the same limits on
patterns as used in the grep function.

\item[\code{dat}] User can specify a data.frame object that will be used to remove
any variables that are not listed in names(dat. As default it is set as NULL.
In this case, the formula is created simply from the
outcome.name and input.names.

\item[\code{interactions}] A list of character vectors. Each character vector es the
names of the variables that form a single interaction. Specifying
interactions = list(c("x", "y"), c("x", "z"), c("y", "z"), c("x", "y", "z"))
would lead to the interactions x*y + x*z + y*z + x*y*z.

\item[\code{force.main.effects}] This is a logical value. When TRUE, the intent is
that any term ed as an interaction (of multiple variables) must also be
listed individually as a main effect.

\item[\code{reduce}] A logical value. When dat is not NULL and reduce is TRUE,
additional quality checks are performed to examine the input variables.
Any input variables that exhibit a lack of contrast will be excluded from
the model. This search is global by default but may be conducted separately
in subsets of the outcome variables by specifying
max.outcome.categories.to.search. Additionally, any input variables that
exhibit too many contrasts, as defined by max.input.categories, will also
be excluded.

\item[\code{max.input.categories}] Limits the maximum number of variables that will
be employed in the formula. As default it is set at 20, but users can still
change at his/her convenience.

\item[\code{max.outcome.categories.to.search}] A numeric value. The create.formula
function es a feature that identifies input variables exhibiting a lack of
contrast. When reduce = TRUE, these variables are automatically excluded
from the resulting formula. This search may be expanded to subsets of the
outcome when the number of unique measured values of the outcome is no
greater than max.outcome.categories.to.search. In this case, each subset
of the outcome will be separately examined, and any inputs that exhibit a
lack of contrast within at least one subset will be excluded.

\item[\code{order.as}] User can specify the order the input variables in the
formula in a variety of ways for patterns: increasing for increasing alphabet
order, decreasing for decreasing alphabet order, column.order for as they
appear in data, and as.specified for maintaining the user's specified order.

\item[\code{include.backtick}] Add backticks if needed. As default it is set as
'as.needed', which add backticks when only it is needed. The other option is
'all'. The use of include.backtick = "all" is limited to cases in which the
output is generated as a character variable. When the output is generated
as a formula object, then R automatically removes all unnecessary backticks.
That is, it is only compatible when format.as != formula.

\item[\code{format.as}] The data type of the output. If not set as "formula", then a
character vector will be returned.

\item[\code{variables.to.exclude}] A character vector. Any variable specified in
variables.to.exclude will be dropped from the formula, both in the individual
inputs and in any associated interactions. This step supersedes the inclusion
of any variables specified for inclusion in the other parameters.

\item[\code{include.intercept}] A logical value. When FALSE, the intercept will be
removed from the formula.
\end{ldescription}
\end{Arguments}
%
\begin{Details}\relax
Return as the data type of the output.  If not set as "formula",
then a character vector will be returned.
The input.names and names of variables matching the input.patterns will be
concatenated to form the full list of input variables.
\end{Details}
%
\begin{Examples}
\begin{ExampleCode}
 n <- 10
 dd <- data.table::data.table(w = rnorm(n= n), x = rnorm(n = n), pixel_1 = rnorm(n = n))
 dd[, pixel_2 := 0.3 * pixel_1 + rnorm(n)]
 dd[, y := 5 * x + 3 * pixel_1 + 2 * pixel_2 + rnorm(n)]

 create.formula(outcome.name = "y", input.names = "x", input.patterns = c("pi", "xel"), dat = dd)

\end{ExampleCode}
\end{Examples}
\inputencoding{utf8}
\HeaderA{reduce.existing.formula}{Reduce Existing Formula}{reduce.existing.formula}
%
\begin{Description}\relax
The reduce.existing.formula function was designed to perform quality checks
and automatic removal of impractical variables can also be accessed when an
existing formula has been previously constructed. This method uses natural
language processing techniques to deconstruct the components of a formula.
\end{Description}
%
\begin{Usage}
\begin{verbatim}
reduce.existing.formula(
  the.initial.formula,
  dat,
  max.input.categories = 20,
  max.outcome.categories.to.search = 4,
  force.main.effects = TRUE,
  order.as = "as.specified",
  include.backtick = "as.needed",
  format.as = "formula"
)
\end{verbatim}
\end{Usage}
%
\begin{Arguments}
\begin{ldescription}
\item[\code{the.initial.formula}] is an object of class "formula" or "character"
that states the inputs and output in the form y \textasciitilde{} x1 + x2.

\item[\code{dat}] Data frame, list or environment (or object coercible by
as.data.frame to a data frame) containing the variables in the model.

\item[\code{max.input.categories}] Limits the maximum number of variables that will
be employed in the formula.As default it is set at 20, but users can still
change at his/her convenience.

\item[\code{max.outcome.categories.to.search}] A numeric value. The create.formula
function es a feature that identifies input variables exhibiting a lack of
contrast. When reduce = TRUE, these variables are automatically excluded from
the resulting formula. This search may be expanded to subsets of the outcome
when the number of unique measured values of the outcome is no greater than
max.outcome.categories.to.search. In this case, each subset of the outcome
will be separately examined, and any inputs builthat exhibit a lack of
contrast within at least one subset will be excluded.

\item[\code{force.main.effects}] This is a logical value.  When TRUE, the intent is
that any term ed as an interaction (of multiple variables) must also be
listed individually as a main effect.

\item[\code{order.as}] rearranges its first argument into ascending or descending
order.

\item[\code{include.backtick}] Add backticks to make a appropriate variable

\item[\code{format.as}] The data type of the output.  If not set as "formula", then
a character vector will be returned.
\end{ldescription}
\end{Arguments}
%
\begin{Examples}
\begin{ExampleCode}
data('snack.dat')
the.initial.formula <- Awareness ~ .

reduce.existing.formula(the.initial.formula = the.initial.formula,dat = snack.dat,
  max.input.categories = 30)$formula

\end{ExampleCode}
\end{Examples}
\inputencoding{utf8}
\HeaderA{snack.dat}{Simulated Marketing Data}{snack.dat}
\keyword{datasets}{snack.dat}
%
\begin{Description}\relax
contains information from the (fictionalized) marketing survey’s data.
\end{Description}
%
\begin{Usage}
\begin{verbatim}
snack.dat
\end{verbatim}
\end{Usage}
%
\begin{Format}
A data frame of 23000 rows and 23 columns
\begin{description}

\item[User ID] Character values assigning a unique customer value
\item[Age] Numeric values displaying the age of customer in years
\item[Gender] Character value describing gender of the customer
\item[Income] Numeric values displaying the income of the customer
\item[Region] Numeric values describing the region of the customer
\item[Persona] Character value describing the customer persona:"Millennial Muncher" "Righteous Reviewer" "Mainstream Maynard" "Savvy Samantha" "Easygoing Edith" "Old School Oliver"
\item[Product] Character value describing product consumed by the customer
\item[Awareness] Numeric values displaying the customer awareness level
\item[BP\_For\_Me\_0\_10] Numeric value displaying brand perception survey result scale (0-10)
\item[BP\_Fits\_Budget\_0\_10] Numeric value displaying brand perception survey results for budget scale (0-10)
\item[BP\_Tastes\_Great\_0\_10] Numeric value displaying brand perception survey results for tastes scale (0-10)
\item[BP\_Good\_To\_Share\_0\_10] Numeric value displaying brand perception survey results for good to share scale (0-10)
\item[BP\_Like\_Logo\_0\_10] Numeric value displaying brand perception survey results for like logo scale (0-10)
\item[BP\_Special\_Occasions\_0\_10] Numeric value displaying brand perception survey results for special occasion scale (0-10)
\item[BP\_Everyday\_Snack\_0\_10] Numeric value displaying brand perception survey results for everyday snack scale (0-10)
\item[BP\_Healthy\_0\_10] Numeric value displaying brand perception survey results for healthy scale (0-10)
\item[BP\_Delicious\_0\_10] Numeric value displaying brand perception survey results for delicious scale (0-10)
\item[BP\_Right\_Amount\_0\_10] Numeric value displaying brand perception survey results for right amount scale (0-10)
\item[BP\_Relaxing\_0\_10] Numeric value displaying brand perception survey results for relaxing scale (0-10)
\item[Consideration] Numeric displaying if the customer would consider this product 1: Yes, 0: No
\item[Consumption] Numeric displaying if the customer would consume this product 1: Yes, 0: No
\item[Satisfaction] Numeric displaying if the customer was satisfied by this product 1: Yes, 0: No
\item[Advocacy] Numeric displaying if the customer would advocate for this product 1: Yes, 0: No
\item[Age Group] Categorical variable that breaks the Users into 4 different groups
\item[Income Group] Categorical variable that breaks the Users into 5 different levels

\end{description}

\end{Format}
%
\begin{Source}\relax
"Randomly generated data"
\end{Source}
\printindex{}
\end{document}
